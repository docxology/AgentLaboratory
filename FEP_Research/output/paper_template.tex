\documentclass[12pt,a4paper]{article}
\usepackage[utf8]{inputenc}
\usepackage{amsmath,amssymb,amsfonts}
\usepackage{graphicx}
\usepackage{hyperref}
\usepackage{natbib}
\usepackage{booktabs}
\usepackage{xcolor}
\usepackage{geometry}

\geometry{margin=1in}
\hypersetup{colorlinks=true, linkcolor=blue, citecolor=blue, urlcolor=blue}

\title{The Free Energy Principle: A Review and Synthesis}
\author{Agent Laboratory}
\date{\today}

\begin{document}

\maketitle

\begin{abstract}
This paper provides a concise review and synthesis of the Free Energy Principle (FEP), a theoretical framework proposed to unify our understanding of brain function and behavior. We review two key papers on the topic, exploring the core mathematical foundations, theoretical underpinnings, and applications of the FEP in cognitive science and neuroscience. The review aims to make the principle accessible to readers with basic knowledge of computational neuroscience while highlighting its strengths, limitations, and future directions.
\end{abstract}

\section{Introduction}
% Brief overview of the Free Energy Principle
% Historical context and development
% Importance in cognitive science and neuroscience
% Outline of the paper

\section{Theoretical Foundations}
% Bayesian brain hypothesis
% Information theory and thermodynamics connections
% Relation to other theoretical frameworks (e.g., predictive coding)

\section{Mathematical Formulations}
% Core mathematical expressions
% Variational free energy
% Bayesian inference and model evidence
% Simplified explanations with relevant equations

\section{Key Concepts}
\subsection{Active Inference}
% How action can be understood as minimizing expected free energy

\subsection{Predictive Coding}
% Hierarchical predictive processing
% Error minimization processes

\subsection{Self-organization and Autopoiesis}
% Connection to biological self-organization
% Homeostasis and allostasis

\section{Applications}
\subsection{Cognitive Neuroscience}
% Applications in understanding perception, action, and learning

\subsection{Computational Psychiatry}
% Understanding psychopathology through the FEP lens

\subsection{Artificial Intelligence}
% Implications for AI development

\section{Critiques and Limitations}
% Theoretical challenges
% Empirical challenges
% Alternative perspectives

\section{Future Directions}
% Emerging research questions
% Potential extensions of the theory

\section{Conclusion}
% Summary of key points
% Broader significance

\bibliographystyle{apalike}
\bibliography{references}

\end{document} 