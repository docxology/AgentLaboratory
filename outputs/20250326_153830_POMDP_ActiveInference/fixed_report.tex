
\documentclass[11pt,a4paper]{article}

% Basic packages
\usepackage[utf8]{inputenc}
\usepackage[T1]{fontenc}
\usepackage{lmodern}
\usepackage{microtype}
\usepackage{graphicx}
\usepackage{booktabs}
\usepackage{array}
\usepackage{parskip}
\usepackage{xcolor}
\usepackage{hyperref}
\usepackage{listings}
\usepackage{float}
\usepackage{soul}
\usepackage{amsmath}
\usepackage{amssymb}
\usepackage{mathtools}
% Try to only include algorithm packages if available
\IfFileExists{algorithm.sty}{
    \usepackage{algorithm}
    \usepackage{algpseudocode}
}{
    % Algorithm package not available, skip it
}
\usepackage{tikz}
\usetikzlibrary{shapes,arrows,positioning}

% Page setup
\usepackage[margin=1in]{geometry}

% Header and footer
\usepackage{fancyhdr}
\pagestyle{fancy}
\fancyhf{}
\renewcommand{\headrulewidth}{0.4pt}
\renewcommand{\footrulewidth}{0.4pt}
\fancyhead[L]{Agent Laboratory}
\fancyhead[R]{\today}
\fancyfoot[C]{\thepage}

% Custom colors
\definecolor{codegreen}{rgb}{0,0.6,0}
\definecolor{codegray}{rgb}{0.5,0.5,0.5}
\definecolor{codepurple}{rgb}{0.58,0,0.82}
\definecolor{backcolour}{rgb}{0.95,0.95,0.95}

% Code listing style
\lstdefinestyle{mystyle}{
    backgroundcolor=\color{backcolour},   
    commentstyle=\color{codegreen},
    keywordstyle=\color{magenta},
    numberstyle=\tiny\color{codegray},
    stringstyle=\color{codepurple},
    basicstyle=\ttfamily\footnotesize,
    breakatwhitespace=false,         
    breaklines=true,                 
    captionpos=b,                    
    keepspaces=true,                 
    numbers=left,                    
    numbersep=5pt,                  
    showspaces=false,                
    showstringspaces=false,
    showtabs=false,                  
    tabsize=2
}
\lstset{style=mystyle}

% Title
\title{\textbf{\Large{Research Report:}} \\ \huge{\textsf{POMDP Implementation with Active Inference}}}
\author{Agent Laboratory Team}
\date{\today}

\begin{document}

\maketitle

\begin{abstract}
This report documents the methodology, experiments, and findings of research conducted using the Agent Laboratory framework. The focus of this research is on \textbf{POMDP Implementation with Active Inference}.

The research was conducted through a systematic process involving multiple agent collaborations, including professors, engineers, and critics, to ensure comprehensive and robust results.
\end{abstract}

\tableofcontents
\newpage

\section{Introduction}
This research addresses POMDP Implementation with Active Inference. The work was conducted using a systematic process implemented within the Agent Laboratory framework, involving multiple phases of research, development, and analysis.

The research followed these key phases:
\begin{itemize}

\item \textbf{Literature Review} - Review of existing research and methodologies
\item \textbf{Implementation} - Development and implementation of algorithms and models
\item \textbf{Evaluation} - Testing and evaluating the implemented solution
\item \textbf{Analysis} - Analysis of results and drawing conclusions

\end{itemize}

Each phase was approached collaboratively by multiple expert agents, including research professors, engineers, and critics, to ensure comprehensive, rigorous, and technically sound results.


\section{Agent Discourse and Collaboration}

The research process involved collaboration between multiple expert agents, each contributing their specific expertise to enhance the quality and rigor of the work.

\textit{No agent discourse available.}


\section{Conclusion}

This report has documented the comprehensive research process for POMDP Implementation with Active Inference. Through systematic collaboration between expert agents, including professors, engineers, and critics, the research progressed through multiple phases from initial planning to final implementation and analysis.

The key contributions include:
\begin{itemize}
    \item A systematic methodology for approaching POMDP Implementation with Active Inference
    \item Technical implementation demonstrating the principles in action
    \item Critical analysis of results and implications
    \item Insights for future research directions
\end{itemize}

The Agent Laboratory framework has facilitated this multi-agent, multi-phase research process, enabling structured collaboration and comprehensive documentation throughout.

\end{document}
